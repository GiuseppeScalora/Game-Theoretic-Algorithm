\documentclass[conference]{IEEEtran}
\IEEEoverridecommandlockouts
% The preceding line is only needed to identify funding in the first footnote. If that is unneeded, please comment it out.
\usepackage{cite}
\usepackage{amsmath,amssymb,amsfonts}
\usepackage{algorithmic}
\usepackage{graphicx}
\usepackage{textcomp}
\usepackage{xcolor}
\usepackage{amsmath}
\usepackage{amssymb}
\def\BibTeX{{\rm B\kern-.05em{\sc i\kern-.025em b}\kern-.08em
    T\kern-.1667em\lower.7ex\hbox{E}\kern-.125emX}}
\begin{document}

\title{Game Theory on High-Level-Synthesis\\
}

\author{\IEEEauthorblockN{ Giuseppe Scalora}
\IEEEauthorblockA{\textit{Electronic Engineering} \\
\textit{Hochschule Hamm-lippstadt}\\
Lippstadt, Germany \\
giuseppe.scalora@stud.hshl.de}

}

\maketitle

\begin{abstract}
This document is a model and instructions for \LaTeX.
This and the IEEEtran.cls file define the components of your paper [title, text, heads, etc.]. *CRITICAL: Do Not Use Symbols, Special Characters, Footnotes, 
or Math in Paper Title or Abstract.
\end{abstract}

\begin{IEEEkeywords}
component, formatting, style, styling, insert
\end{IEEEkeywords}

\section{Introduction}
This document is a model and instructions for \LaTeX.
Please observe the conference page limits. 

\section{Background of game theory}
The very beginning of some ideas regarding the game theory seem to be assigned to the 18th century while the actual and first development began in the early years of the 20th century (around 1920s) with the work of the mathematicians Emile Borel (1871–1956) and John von Neumann (1903–57). The latter mentioned, published a first work on game theory, alongside Oskar Morgenstein, called "Theory of games and economic behaviour". In the 1950s game-theoretic models began to be used in economic theory, political science and also by psychologists who began studying how human subjects behave in experimental games. Later during the 70s the theory started taking place and being applied into many others fields, one of this recently included, which is computer science[1] (introduction to game theory reference).

\section{The key concepts behind the theory}
In order to understand how and why this algorithm has been developed it is necessary to get familiar with some important concepts used as means of understanding for everyone.
\subsection{Static games}
Let us take into account an interactive decision problem which involves two or more individuals who have to come to a final decision according to which the payoff for each person depends on the other individual decision. Commonly, such decision making problems can be assessed as "games" and the individuals making the decisions (2 or more) are called "players". The game does not necessarily need to have a winner and a loser but it can have restricted features, which will lead us to call such games as recreational. On the other hand, games
that have winners and losers are called zero-sum games (which will not be discussed in this matter). 
Therefore the definition of a static game describes games in which the decisions are made simultaneously by the players and in ignorance of choices made by other players in the game. According to this they can be referred also as simultaneous decision games, simply because the order in which the decision must be taken is irrelevant[](game theory webb J.). A technical description of the main points a static game looks as follows:
\begin{enumerate}
  \item A set of players must be defined in the form of a set: \[i\in       \{1,2,3...\}\]
  \item A pure strategy set for all players, \textbf{S{\textsubscript{i}}}
  \item Players payoffs, according to the decisions combinations.
\end{enumerate} 
A very popular example of such games is shown and explained in the following paragraph: "the prisoner´s dilemma".
\subsubsection{The prisoner's dilemma}
Let us assume two different prisoners are going through a trial for a crime, whether it has occured for real or not. Both prisoners have a degree of freedom limited to 2 possible answers, confessing or remain silent, therefore the number of possible outcomes is bounded to 4. These, respectively, are: scenario 1 (For simplification prisoner 1 and 2 will be assessed by using simply P1 and P2). P1 confesses and P2 confesses, this outcome leads both prisoners to be sentenced to 4 years of prison. Scenario 2. P1 confesses and P2 remains silent which leads to P1 being sentenced to 1 year of prison while P2 is sentenced to 5 years. Scenario 3. P1 remains silent and P2 confesses, this will result in the opposite of what just mentioned, P1 will get 5 years and P2 only 1 year. Scenario 4. P1 and P2 remain silent, this will lead to both being sentenced to only 1 year. For a better understanding, the following matrix will visually show the concept in a more schematic way.
\centerline{\includegraphics[scale=0.5]{matrix.png}}  
\centerline{\textit{Image 1 - prisoner's dilemma matrix}}

Now as you can see from this 2X2 matrix, the most stable outcome would be the scenario 1, in which both prisoners confess the crime and are not, obviously, allowed to change their declaration at any given time. Meanwhile within the rest of the scenarios the prisoners might decide, at any point, to change their mind in order to obtain a better outcome for themselves, more on the aspect of being egoistic[2](algorithmic game theory reference).


\subsection{Nash equilibrium}
When dealing with game theory it is necessary to create and instantiate some equilibrium that helps reach the best possible outcome for a system of any nature. When talking about Nash equilibrium, it is only taken into consideration a non-cooperative game which involves 2 or more players. In this particular case, each player's decision should depend on the other players' decisions, therefore each of them must give an assumption or, better, create a belief of what the others will choose. The mainly basis for this derives from the assumption that the players have already gotten a past experience in the game in order to create reliable beliefs about the other players' possible behaviours.
This is only possible in idealized circumnstances, although in many cases the players, even with past experiences, are only aware of the behaviours of other typical players but not of specific ones. This means that in the solution, given by the Nash equilibrium each player's beliefs are assumed to be correct hence the player's choices will be made according to other players' choices. 
\textit{"A Nash equilibrium is an action profile a\textsuperscript{\text{*}} with the property that no player \textbf{i} can do better by choosing an action different from a\textsuperscript{\text{*}}\textsubscript{i}, given that every other player \textbf{j} adheres to a\textsuperscript{\text{*}}\textsubscript{j} }
(martinj osborne an introduction to game theory-oxford university press usa2003 page 32) .

An excellent style manual for science writers is \cite{b7}.

\subsection{Dynamic games}
On the contrary of static games, dynamic games involve many situations of interest in which decisions are
made at various times, not simultaneously, taking into account the choices made earlier. Therefore dynamic games introduce an explicit time-schedule which describes the exact time spans, or time points, during which the players make their decisions.
Dynamic games can be represented by a game decision tree. The filled black circles show the time points at which decisions are taken. The connections between circles are self explanatory, they represent branches from which other action could occur. At the bottom of the tree, after every decision sequence has come to its end, there is the payoff, usually specified and written. By convention the tree is drawn "upside-down" which means that the time increases as the tree branches downwards. (reference: game theory Webb book)
\subsubsection{Dinner party game}
In order to understand the example, the tree will be first shown and therefore analyzed.\\
\centerline{\includegraphics[scale=0.5]{gametree.png}}  
\centerline{\textit{Image 2 - dinner party game tree}}
Let us consider the tree shown, in which the Husband will start the tree of decisions by buying either meat (M) or fish (F). Meanwhile the Wife will buy either Red wine (R) or White wine (W). We assume that both wife and husband prefer to drink red wine with meat and white wine with fish. Although the husband prefers to eat meat for dinner and the wife prefers the opposite, the fish. The possible payoffs will be stated as following:\\ \\
\centerline{$\pi$\textsubscript{h} (M, R) = 2 ,  $\pi$\textsubscript{h} (F, W) = 1} \\
\centerline{$\pi$\textsubscript{h} (F, R) = $\pi$\textsubscript{h} (M, W) = 0} \\\\
\centerline{$\pi$\textsubscript{w} (M, R) = 1 ,  $\pi$\textsubscript{w} (F, W) = 2} \\
\centerline{$\pi$\textsubscript{w} (F, R) = $\pi$\textsubscript{w} (M, W) = 0}
\\

In order to solve this problem, we can assume that the husband has told the wife whether he bought fish or meat. Therefore, the discussion proceeds with a backward induction approach, if the husband has bought meat, for instance, then the wife will be aware of that and hence she will buy red wine in order to have a payoff = 1 rather than a payoff = 0 buying white wine. If the husband, instead, has bought fish, then the wife will proceed with buying white wine and have a payoff = 2 rather than = 0 with red wine and fish.
It is pretty clear that in the best possible scenario, the husband will buy meat, in order to have a payoff = 2 and this will lead the wife in choosing red wine (payoff = 1) for the wife. In the end the dinner will be made of meat and red wine.




\subsection{Stackelberg games}
Another important concept, which will be helpful to understand the applications of game theory on specific fields, is the "Stackelberg games model".
We suppose two firms are competing with each other, producing the same good. Firm \textit{i}`s amount of product to produce is \textit{q\textsubscript{i}} and the total output production cost, for all the units, will be assessed as C\textsubscript{i}, meanwhile the selling price qill be referred as P\textsubscript{d}. According to a Stackelberg model, the two players (firms in this case) make their decisions sequentially, therefore, if firm i=1 makes the decision first, the second firm will follow knowing the decision of the first firm.\\
Given these initial information:
\begin{enumerate}
\item Set of 2 players (firms), \textit{i}
\item Set of sequences (q\textsubscript{1}, q\textsubscript{2} of firms outputs.
\item 
\end{enumerate}

 we can calculate the payoff for each firm which will look as such:\\\\
\centerline{q\textsubscript{i}P(q\textsubscript{1} + q\textsubscript{2}) - C\textsubscript{i}(q\textsubscript{i}) for i = 1,2}\\\\

\section{Applications}
\paragraph{Positioning Figures and Tables} Place figures and tables at the top and 
bottom of columns. Avoid placing them in the middle of columns. Large 
figures and tables may span across both columns. Figure captions should be 
below the figures; table heads should appear above the tables. Insert 
figures and tables after they are cited in the text. Use the abbreviation 
``Fig.~\ref{fig}'', even at the beginning of a sentence.

\begin{table}[htbp]
\caption{Table Type Styles}
\begin{center}
\begin{tabular}{|c|c|c|c|}
\hline
\textbf{Table}&\multicolumn{3}{|c|}{\textbf{Table Column Head}} \\
\cline{2-4} 
\textbf{Head} & \textbf{\textit{Table column subhead}}& \textbf{\textit{Subhead}}& \textbf{\textit{Subhead}} \\
\hline
copy& More table copy$^{\mathrm{a}}$& &  \\
\hline
\multicolumn{4}{l}{$^{\mathrm{a}}$Sample of a Table footnote.}
\end{tabular}
\label{tab1}
\end{center}
\end{table}

\begin{figure}[htbp]
\centerline{\includegraphics{fig1.png}}
\caption{Example of a figure caption.}
\label{fig}
\end{figure}

Figure Labels: Use 8 point Times New Roman for Figure labels. Use words 
rather than symbols or abbreviations when writing Figure axis labels to 
avoid confusing the reader. As an example, write the quantity 
``Magnetization'', or ``Magnetization, M'', not just ``M''. If including 
units in the label, present them within parentheses. Do not label axes only 
with units. In the example, write ``Magnetization (A/m)'' or ``Magnetization 
\{A[m(1)]\}'', not just ``A/m''. Do not label axes with a ratio of 
quantities and units. For example, write ``Temperature (K)'', not 
``Temperature/K''.

\section*{Acknowledgment}

The preferred spelling of the word ``acknowledgment'' in America is without 
an ``e'' after the ``g''. Avoid the stilted expression ``one of us (R. B. 
G.) thanks $\ldots$''. Instead, try ``R. B. G. thanks$\ldots$''. Put sponsor 
acknowledgments in the unnumbered footnote on the first page.

\section*{References}

Please number citations consecutively within brackets \cite{b1}. The 
sentence punctuation follows the bracket \cite{b2}. Refer simply to the reference 
number, as in \cite{b3}---do not use ``Ref. \cite{b3}'' or ``reference \cite{b3}'' except at 
the beginning of a sentence: ``Reference \cite{b3} was the first $\ldots$''

Number footnotes separately in superscripts. Place the actual footnote at 
the bottom of the column in which it was cited. Do not put footnotes in the 
abstract or reference list. Use letters for table footnotes.

Unless there are six authors or more give all authors' names; do not use 
``et al.''. Papers that have not been published, even if they have been 
submitted for publication, should be cited as ``unpublished'' \cite{b4}. Papers 
that have been accepted for publication should be cited as ``in press'' \cite{b5}. 
Capitalize only the first word in a paper title, except for proper nouns and 
element symbols.

For papers published in translation journals, please give the English 
citation first, followed by the original foreign-language citation \cite{b6}.

\begin{thebibliography}{00}
\bibitem{b1} G. Eason, B. Noble, and I. N. Sneddon, ``On certain integrals of Lipschitz-Hankel type involving products of Bessel functions,'' Phil. Trans. Roy. Soc. London, vol. A247, pp. 529--551, April 1955.
\bibitem{b2} J. Clerk Maxwell, A Treatise on Electricity and Magnetism, 3rd ed., vol. 2. Oxford: Clarendon, 1892, pp.68--73.
\bibitem{b3} I. S. Jacobs and C. P. Bean, ``Fine particles, thin films and exchange anisotropy,'' in Magnetism, vol. III, G. T. Rado and H. Suhl, Eds. New York: Academic, 1963, pp. 271--350.
\bibitem{b4} K. Elissa, ``Title of paper if known,'' unpublished.
\bibitem{b5} R. Nicole, ``Title of paper with only first word capitalized,'' J. Name Stand. Abbrev., in press.
\bibitem{b6} Y. Yorozu, M. Hirano, K. Oka, and Y. Tagawa, ``Electron spectroscopy studies on magneto-optical media and plastic substrate interface,'' IEEE Transl. J. Magn. Japan, vol. 2, pp. 740--741, August 1987 [Digests 9th Annual Conf. Magnetics Japan, p. 301, 1982].
\bibitem{b7} M. Young, The Technical Writer's Handbook. Mill Valley, CA: University Science, 1989.
\end{thebibliography}
\vspace{12pt}
\color{red}
IEEE conference templates contain guidance text for composing and formatting conference papers. Please ensure that all template text is removed from your conference paper prior to submission to the conference. Failure to remove the template text from your paper may result in your paper not being published.

\end{document}
